% ****** Start of file aipsamp.tex ******
%
%   This file is part of the AIP files in the AIP distribution for REVTeX 4.
%   Version 4.1 of REVTeX, October 2009
%
%   Copyright (c) 2009 American Institute of Physics.
%
%   See the AIP README file for restrictions and more information.
%
% TeX'ing this file requires that you have AMS-LaTeX 2.0 installed
% as well as the rest of the prerequisites for REVTeX 4.1
%
% It also requires running BibTeX. The commands are as follows:
%
%  1)  latex  aipsamp
%  2)  bibtex aipsamp
%  3)  latex  aipsamp
%  4)  latex  aipsamp
%
% Use this file as a source of example code for your aip document.
% Use the file aiptemplate.tex as a template for your document.
\documentclass[%
 aip,
 jmp,%
 amsmath,amssymb,
%preprint,%
 reprint,%
%author-year,%
%author-numerical,%
]{revtex4-1}

\usepackage{graphicx}% Include figure files
\usepackage{dcolumn}% Align table columns on decimal point
\usepackage{bm}% bold math
%\usepackage[mathlines]{lineno}% Enable numbering of text and display math
%\linenumbers\relax % Commence numbering lines

\begin{document}

\preprint{AIP/123-QED}

\title[complex networks gradus]{Complex networks gradus ad parnassum}% Force line breaks with \\
%\thanks{Footnote to title of article.}

\author{R. Fabbri}
 \altaffiliation[Also at ]{S\~ao Carlos Physics Institute of Physics, University of S\~ao Paulo.}%Lines break automatically or can be forced with \\
%\author{B. Author}%
% \email{Second.Author@institution.edu.}
%\affiliation{ 
%Authors' institution and/or address%\\This line break forced with \textbackslash\textbackslash
%}%
%
%\author{C. Author}
% \homepage{http://www.Second.institution.edu/~Charlie.Author.}
%\affiliation{%
%Second institution and/or address%\\This line break forced% with \\
%}%

\date{\today}% It is always \today, today,
             %  but any date may be explicitly specified

\begin{abstract}
Complex networks have received much attention from the academic community in the past decade,
with impacts in both science and society.
Even so, comprehensive guides are usually lengthy and unaccessible to non-specialists.
This text presents the subject and vocabulary issues in literature, the paradigmatic models and
their typical measures. A discussion about the ubiquity of network structures, and our own
existence as networks, should ease the reader to grasp essential and useful concepts.
Metrics, software, related work and exercises are in the Appendixes.
\end{abstract}

\pacs{01.30.Rr, 05.65.+b, 89.75.Kd, 89.75.Fb}% PACS, the Physics and Astronomy
                             % Classification Scheme.
\keywords{complex networks, statistical physics, tutorial}%Use showkeys class option if keyword
                              %display desired
\maketitle

%\begin{quotation}
%The ``lead paragraph'' is encapsulated with the \LaTeX\ 
%\verb+quotation+ environment and is formatted as a single paragraph before the first section heading. 
%(The \verb+quotation+ environment reverts to its usual meaning after the first sectioning command.) 
%Note that numbered references are allowed in the lead paragraph.
%%
%The lead paragraph will only be found in an article being prepared for the journal \textit{Chaos}.
%\end{quotation}

\section{\label{sec:intro}Introduction}
\subsection{Basic concepts}
\subsubsection{Graph}
\subsubsection{Complex networks}
\subsection{Jargon synonyms and ambiguities}
Transitivity, clustering, connectivity, hubs, authorities, intermediaries (betweenness and Erd\"os Sector), periphery related to diameter of the connective sector, center/hubs. Complexity, Complex Systems and Complex Networks. Anthropological field vs influence.

\section{\label{sec:models}Paradigmatic models}
\subsection{Small-world}
Characteristics, Measures, Generative models.
\subsection{Geographic}
Characteristics, Measures, Generative models.
\subsection{Scale-Free}
Characteristics, Measures, Generative models.
\subsection{Erd\"os-R\'enyi}
Characteristics, Measures, Generative models.
\subsection{Other recurrent network characteristics}
\section{\label{sec:inet}You-networks, I-networks}
\subsection{\label{sec:stabdiff}Stability and differentiation in human social networks}
\subsection{\label{sec:harn}Harnessing}
Procedures, ontologies, data, software, art, anthropological physics, 
future forecast, next steps and future work.

\section{\label{sec:conc}Conclusions}

\begin{acknowledgments}
We wish to acknowledge the support of 
\end{acknowledgments}

\appendix

\section{Measures}
Equation, definition and reference.

\section{Software, data and media}
\section{Related works}
\subsection{Articles}
\subsection{Books}
\section{Exercises}
Theory, real data analysis, anthropological physics experiments.

%\begin{subequations}
%\begin{eqnarray}
%E&=&mc, \label{appa}
%\\
%E&=&mc^2, \label{appb}
%\\
%E&\agt& mc^3. \label{appc}
%\end{eqnarray}
%\end{subequations}

\nocite{*}
\bibliography{aipsamp}% Produces the bibliography via BibTeX.

\end{document}
%
% ****** End of file aipsamp.tex ******
